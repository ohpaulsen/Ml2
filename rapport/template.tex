\documentclass[a4paper,12pt,pdftex]{article}

\usepackage{fancyhdr}
\usepackage{parskip}
\usepackage[utf8]{inputenc}
\usepackage{hyperref}

\newcommand{\HRule}{\rule{\linewidth}{0.2mm}}

\def\name{Ole Henrik Paulsen}
\def\studentnumber{130572}
\def\course{Machine Learning and Pattern Recognition | IMT4612}
\def\school{Gjøvik University Collage}
\def\reportname{Assigment 2}
% Spring, 2013 - Autumn, 2010 - Both are valid
\def\semester{Spring 2014}

\hypersetup
{
    pdftitle={\reportname},
    pdfauthor={\name},
    pdfsubject={\reportname},
    colorlinks=true,
    linkcolor=blue,
    citecolor=blue,
    urlcolor=blue   
}

\fancypagestyle{titlefooter}
{                                                                               
    \fancyhf{}                                                                  
    \fancyfoot[c]{\footnotesize This document was compiled on \today}           
} 

\begin{document}

\begin{titlepage}                                                               
    \begin{center}                                                                                                   
                                                                                
        \large \course\\                                                        
        \large \school, \semester\\[0.4cm]                                       
        \HRule\\[1.5cm]                                                         
                                                                                
        \begin{minipage}{0.4\textwidth}                                         
            \begin{flushleft}                                                   
                % Some lecturers does not want to know your name...
                \small \emph{Author:} \name\\                                  
                \small \emph{Student number:} \studentnumber\\                  
            \end{flushleft}                                                     
        \end{minipage}                                                          
                                                                                
        \ \\[7.0cm]                                                             
        \LARGE\textbf{\reportname}                                              
                                                                                
    \end{center}                                                                
    \thispagestyle{titlefooter}                                                 
\end{titlepage}                                                                 
                                                                                
\tableofcontents                                                                
\clearpage   

\begin{abstract}

    Assigment 2 in Machinelearning

\end{abstract}

\section{Learning as a Search}
\subsection{Global optimal solutions}
\subsection{Genetic Algorithm(GA)}
\subsection{Gradient Descent method}
\subsection{Performance domain}

\section{Statistical Learning}
\subsection{Computer program}
I did the programming in Python with the library Numpy and Time. You need to install Numpy, but time is a core
library of Python. Numpy are used to handle sqrt, max, min and mathematical functions on arrays. I also used Numpy
to read in data from the txt files. Numpy also have functions for Euclidean and chebyshev, but they are NOT used in my program.

K Neaarest Neighbor are also programmed by from the bottom instead of using a library. 

*K nearst Neighbor code*

The program are scaled to handle large amount of input, both train and validation data, and 13 attributes takes under 3 secounds to handle.

\subsection{Read input files}

The train.txt and validation.txt are read into the program with Numpy's genfromtxt.
*genfromtxt code* 

I save the input data in a masked arrays. and split out the label into a own array.

*Splitting code* 

I confirm the input is 120 train samples and 10 validation samples in the output.

\subsection{Radar and Area plot}
\subsection{Distance Algorythms}

The follow three algorithms are included into the program:

Euclidean:

Squar Euclidean:

Chebyshev:

As you can see they got different output from each other
\subsection{Output}

The output of the program is as follow:

*Output from the program*

\nocite{*}

\bibliographystyle{acm}
\bibliography{references}

\end{document}
